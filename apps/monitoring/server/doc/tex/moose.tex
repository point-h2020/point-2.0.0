\documentclass[a4paper,11pt,titlepage]{report}
\usepackage[printonlyused]{acronym}
\usepackage{amsmath}
\usepackage[UKenglish]{babel}
\usepackage{caption}
\usepackage{color}
\usepackage[hidelinks]{hyperref}
\usepackage{graphicx}
\usepackage[T1]{fontenc}
\usepackage[utf8]{inputenc}
\usepackage{listings}
\usepackage{lmodern}

\title{\textbf{MOnitOring SErver (MOOSE)}}
\author{Sebastian Robitzsch <sebastian.robitzsch@interdigital.com>}

\begin{document}
\setcounter{secnumdepth}{3} %enumerate up to subsubsection (not paragraph)
\lstset{language=C++}   
\maketitle
\tableofcontents
\newpage
\begin{acronym}[TDMA]
	\acro{CID}{Content Identifer}
	\acro{CMC}{Co-incidental Multicast}
	\acro{cNAP}{client-side \acl{NAP}}
	\acro{eNAP}{extended \acl{NAP}}
	\acro{EUI-48}{48-bit Extended Unique Identifier}
	\acro{FD}{File Descriptor}
	\acro{FQDN}{Full Qualified Domain Name}
	\acro{GW}{Gateway}
	\acro{HTTP}{Hypertext Transfer Protocol}
	\acro{ICN}{Information-centric Networking}
	\acro{ID}{Identifier}
	\acro{FID}{Forwarding Identifier}
	\acro{LTP}{Lightweight Transport Protocol}
	\acro{MAC}{Medium Access Control}
	\acro{MITU}{Maximum ICN Transmission Unit}
	\acro{MOLY}{Monitoring Library}
	\acro{MTU}{Maximum Transmission Unit}
	\acro{NACK}{Negative Acknowledgement}
	\acro{NAP}{Network Attachment Point}
	\acro{NID}{Node Identifier}
	\acro{PID}{Port Identifier}
	\acro{POINT}{iP Over IcN - the betTer ip}
	\acro{rCID}{reverse \acl{CID}}
	\acro{RTT}{Round Trip Time}
	\acro{RV}{Rendezvous}
	\acro{SA}{Surrogate Agent}
	\acro{SDN}{Software-defined Networking}
	\acro{SE}{Session End}
	\acro{SED}{Session Ended}
	\acro{SK}{Session Key}
	\acro{sNAP}{sever-side \acl{NAP}}
	\acro{STL}{Standard Template Library}
	\acro{TCP}{Transport Control Protocol}
	\acro{UE}{User Equipment}
	\acro{URL}{Uniform Resource Locator}
	\acro{UTP}{Unreliable Transport Protocol}
	\acro{WE}{Window End}
	\acro{WED}{Window Ended}
	\acro{WU}{Window Update}
	\acro{WUD}{Window Updated}
\end{acronym}

\acresetall
%%%%%%%%%%%%%%%%%%%%%%%%%%%%%%%%%%%%%%%%%%%%%%%%%%%%%%%%%%%%%
%%%%%%%%%%%%%%%%%%%%%%%%%%%%%%%%%%%%%%%%%%%%%%%%%%%%%%%%%%%%%
% CHAPTER
%%%%%%%%%%%%%%%%%%%%%%%%%%%%%%%%%%%%%%%%%%%%%%%%%%%%%%%%%%%%%
%%%%%%%%%%%%%%%%%%%%%%%%%%%%%%%%%%%%%%%%%%%%%%%%%%%%%%%%%%%%%
\chapter{Introduction}\label{ch:Introduction}
This document should be seen as some sort of \ac{NAP} documentation to bridge the gap between the POINT deliverables and the code documentation available through Doxygen \cite{Heesch}. 
%%%%%%%%%%%%%%%%%%%%%%%%%%%%%%%%%%%%%%%%%%%%%%%%%%%%%%%%%%%%%
% Section
%%%%%%%%%%%%%%%%%%%%%%%%%%%%%%%%%%%%%%%%%%%%%%%%%%%%%%%%%%%%%
\section{Installation and Configuration}
The compilation and installation of \ac{MOOSE} has been tested on Debian-based Linux distributions. To compile and install \ac{MOOSE} simply use the make file provisioned with the source code:

\begin{lstlisting}
	~$ cd ~/blackadder/apps/monitoring/server
	~$ make
	~$ sudo make install
\end{lstlisting}

The installation copies a default libconfig-based template configuration file \texttt{moose.cfg} to \texttt{/etc/moose} which is the default location for \ac{MOOSE}. All parameters which are not commented are mandatory and must be provided. The variables of the array \texttt{dataPoints} are optional.


\chapter{Work-flows}
monitoringServer-MessageStack.eps
\end{document}
